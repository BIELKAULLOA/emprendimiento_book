% Options for packages loaded elsewhere
\PassOptionsToPackage{unicode}{hyperref}
\PassOptionsToPackage{hyphens}{url}
\PassOptionsToPackage{dvipsnames,svgnames,x11names}{xcolor}
%
\documentclass[
  letterpaper,
  DIV=11,
  numbers=noendperiod]{scrreprt}

\usepackage{amsmath,amssymb}
\usepackage{iftex}
\ifPDFTeX
  \usepackage[T1]{fontenc}
  \usepackage[utf8]{inputenc}
  \usepackage{textcomp} % provide euro and other symbols
\else % if luatex or xetex
  \usepackage{unicode-math}
  \defaultfontfeatures{Scale=MatchLowercase}
  \defaultfontfeatures[\rmfamily]{Ligatures=TeX,Scale=1}
\fi
\usepackage{lmodern}
\ifPDFTeX\else  
    % xetex/luatex font selection
\fi
% Use upquote if available, for straight quotes in verbatim environments
\IfFileExists{upquote.sty}{\usepackage{upquote}}{}
\IfFileExists{microtype.sty}{% use microtype if available
  \usepackage[]{microtype}
  \UseMicrotypeSet[protrusion]{basicmath} % disable protrusion for tt fonts
}{}
\makeatletter
\@ifundefined{KOMAClassName}{% if non-KOMA class
  \IfFileExists{parskip.sty}{%
    \usepackage{parskip}
  }{% else
    \setlength{\parindent}{0pt}
    \setlength{\parskip}{6pt plus 2pt minus 1pt}}
}{% if KOMA class
  \KOMAoptions{parskip=half}}
\makeatother
\usepackage{xcolor}
\setlength{\emergencystretch}{3em} % prevent overfull lines
\setcounter{secnumdepth}{5}
% Make \paragraph and \subparagraph free-standing
\ifx\paragraph\undefined\else
  \let\oldparagraph\paragraph
  \renewcommand{\paragraph}[1]{\oldparagraph{#1}\mbox{}}
\fi
\ifx\subparagraph\undefined\else
  \let\oldsubparagraph\subparagraph
  \renewcommand{\subparagraph}[1]{\oldsubparagraph{#1}\mbox{}}
\fi


\providecommand{\tightlist}{%
  \setlength{\itemsep}{0pt}\setlength{\parskip}{0pt}}\usepackage{longtable,booktabs,array}
\usepackage{calc} % for calculating minipage widths
% Correct order of tables after \paragraph or \subparagraph
\usepackage{etoolbox}
\makeatletter
\patchcmd\longtable{\par}{\if@noskipsec\mbox{}\fi\par}{}{}
\makeatother
% Allow footnotes in longtable head/foot
\IfFileExists{footnotehyper.sty}{\usepackage{footnotehyper}}{\usepackage{footnote}}
\makesavenoteenv{longtable}
\usepackage{graphicx}
\makeatletter
\def\maxwidth{\ifdim\Gin@nat@width>\linewidth\linewidth\else\Gin@nat@width\fi}
\def\maxheight{\ifdim\Gin@nat@height>\textheight\textheight\else\Gin@nat@height\fi}
\makeatother
% Scale images if necessary, so that they will not overflow the page
% margins by default, and it is still possible to overwrite the defaults
% using explicit options in \includegraphics[width, height, ...]{}
\setkeys{Gin}{width=\maxwidth,height=\maxheight,keepaspectratio}
% Set default figure placement to htbp
\makeatletter
\def\fps@figure{htbp}
\makeatother
\newlength{\cslhangindent}
\setlength{\cslhangindent}{1.5em}
\newlength{\csllabelwidth}
\setlength{\csllabelwidth}{3em}
\newlength{\cslentryspacingunit} % times entry-spacing
\setlength{\cslentryspacingunit}{\parskip}
\newenvironment{CSLReferences}[2] % #1 hanging-ident, #2 entry spacing
 {% don't indent paragraphs
  \setlength{\parindent}{0pt}
  % turn on hanging indent if param 1 is 1
  \ifodd #1
  \let\oldpar\par
  \def\par{\hangindent=\cslhangindent\oldpar}
  \fi
  % set entry spacing
  \setlength{\parskip}{#2\cslentryspacingunit}
 }%
 {}
\usepackage{calc}
\newcommand{\CSLBlock}[1]{#1\hfill\break}
\newcommand{\CSLLeftMargin}[1]{\parbox[t]{\csllabelwidth}{#1}}
\newcommand{\CSLRightInline}[1]{\parbox[t]{\linewidth - \csllabelwidth}{#1}\break}
\newcommand{\CSLIndent}[1]{\hspace{\cslhangindent}#1}

\KOMAoption{captions}{tableheading}
\makeatletter
\makeatother
\makeatletter
\@ifpackageloaded{bookmark}{}{\usepackage{bookmark}}
\makeatother
\makeatletter
\@ifpackageloaded{caption}{}{\usepackage{caption}}
\AtBeginDocument{%
\ifdefined\contentsname
  \renewcommand*\contentsname{Table of contents}
\else
  \newcommand\contentsname{Table of contents}
\fi
\ifdefined\listfigurename
  \renewcommand*\listfigurename{List of Figures}
\else
  \newcommand\listfigurename{List of Figures}
\fi
\ifdefined\listtablename
  \renewcommand*\listtablename{List of Tables}
\else
  \newcommand\listtablename{List of Tables}
\fi
\ifdefined\figurename
  \renewcommand*\figurename{Figure}
\else
  \newcommand\figurename{Figure}
\fi
\ifdefined\tablename
  \renewcommand*\tablename{Table}
\else
  \newcommand\tablename{Table}
\fi
}
\@ifpackageloaded{float}{}{\usepackage{float}}
\floatstyle{ruled}
\@ifundefined{c@chapter}{\newfloat{codelisting}{h}{lop}}{\newfloat{codelisting}{h}{lop}[chapter]}
\floatname{codelisting}{Listing}
\newcommand*\listoflistings{\listof{codelisting}{List of Listings}}
\makeatother
\makeatletter
\@ifpackageloaded{caption}{}{\usepackage{caption}}
\@ifpackageloaded{subcaption}{}{\usepackage{subcaption}}
\makeatother
\makeatletter
\@ifpackageloaded{tcolorbox}{}{\usepackage[skins,breakable]{tcolorbox}}
\makeatother
\makeatletter
\@ifundefined{shadecolor}{\definecolor{shadecolor}{rgb}{.97, .97, .97}}
\makeatother
\makeatletter
\makeatother
\makeatletter
\makeatother
\ifLuaTeX
  \usepackage{selnolig}  % disable illegal ligatures
\fi
\IfFileExists{bookmark.sty}{\usepackage{bookmark}}{\usepackage{hyperref}}
\IfFileExists{xurl.sty}{\usepackage{xurl}}{} % add URL line breaks if available
\urlstyle{same} % disable monospaced font for URLs
\hypersetup{
  pdftitle={Emprendimiento con Datos en Turismo},
  pdfauthor={Francisco J. Navarro-Meneses, Editor},
  colorlinks=true,
  linkcolor={blue},
  filecolor={Maroon},
  citecolor={Blue},
  urlcolor={Blue},
  pdfcreator={LaTeX via pandoc}}

\title{Emprendimiento con Datos en Turismo}
\author{Francisco J. Navarro-Meneses, Editor}
\date{2023-06-12}

\begin{document}
\maketitle
\ifdefined\Shaded\renewenvironment{Shaded}{\begin{tcolorbox}[breakable, boxrule=0pt, sharp corners, interior hidden, borderline west={3pt}{0pt}{shadecolor}, enhanced, frame hidden]}{\end{tcolorbox}}\fi

\renewcommand*\contentsname{Table of contents}
{
\hypersetup{linkcolor=}
\setcounter{tocdepth}{2}
\tableofcontents
}
\bookmarksetup{startatroot}

\hypertarget{prefacio}{%
\chapter*{Prefacio}\label{prefacio}}
\addcontentsline{toc}{chapter}{Prefacio}

\markboth{Prefacio}{Prefacio}

Lorem ipsum dolor sit amet, consectetur adipiscing elit. Praesent
dapibus ut libero nec semper. In quam lorem, rutrum in nulla quis,
elementum volutpat odio. Phasellus felis nunc, semper eu sodales eu,
laoreet nec arcu. Sed ac magna quis sapien accumsan gravida. Etiam
tristique dui id elit egestas condimentum. Integer gravida fermentum
placerat. Ut hendrerit viverra ipsum, id vehicula magna tincidunt sed.
Nunc fermentum diam purus, non dignissim purus tincidunt vitae. Sed
tincidunt tortor vitae malesuada molestie. Sed aliquet, est a vulputate
aliquet, massa erat hendrerit arcu, in ultrices nulla nisl vel tortor.
Nam velit est, venenatis et tortor ac, rhoncus feugiat est. Nunc non
neque erat. Etiam eget ipsum fermentum risus lobortis consequat sit amet
sit amet dui. Maecenas auctor vehicula volutpat.

\hypertarget{titular}{%
\section*{Titular}\label{titular}}
\addcontentsline{toc}{section}{Titular}

\markright{Titular}

Etiam ultricies magna imperdiet nunc malesuada, ac lobortis sapien
rhoncus. Aenean eros lectus, accumsan vitae faucibus a, aliquam ac diam.
Maecenas finibus justo non nibh pharetra aliquam. Maecenas egestas, ex
vitae blandit convallis, leo mauris ornare nunc, porta fermentum tellus
est et urna. Aenean eu est et sapien laoreet placerat. Nunc turpis
ipsum, dapibus a ipsum at, tempus pretium lacus. In libero turpis,
tristique id orci a, auctor rhoncus risus. Praesent lacus nunc,
sollicitudin quis ipsum id, pulvinar venenatis mi.

\hypertarget{titular-1}{%
\subsection*{Titular}\label{titular-1}}
\addcontentsline{toc}{subsection}{Titular}

Vivamus libero leo, accumsan non turpis vel, eleifend sodales mauris.
Donec pellentesque, tellus a lacinia vestibulum, nisl lacus dapibus
nibh, id tincidunt turpis erat non nisl. Donec ornare imperdiet metus,
eu sollicitudin risus elementum non. Aliquam metus eros, maximus
dignissim pellentesque nec, venenatis vitae purus. Vivamus laoreet mi
magna, ac malesuada tellus condimentum in. Integer nulla lectus, finibus
sit amet ex nec, finibus molestie diam. Pellentesque gravida ac erat sit
amet tempus.

This is a Quarto book.

\bookmarksetup{startatroot}

\hypertarget{introducciuxf3n}{%
\chapter*{Introducción}\label{introducciuxf3n}}
\addcontentsline{toc}{chapter}{Introducción}

\markboth{Introducción}{Introducción}

\begin{itemize}
\tightlist
\item
  The development of technology and the mounting use of information
  communication technology (ICT) have an important role for the business
  organisations and operations
\item
  Although the need of having a reliable ICT is apparent, there are many
  business organisations that do not adopt this technology. For a large
  business organisation with ample human and financial resources, the
  adoption of ICT may not be a significant problem. However, for
  small-sized business organisations which face resources limitations,
  the ICT adoption becomes a problem.
\item
  This issue turns out to be more severe for small businesses in
  developing countries such as Indonesia, in which its entrepreneurs
  face education and cultural constraints (Anggadwita et al., 2015;
  Tambunan, 2011)
\item
  This is not a book about studying the adoption of ICT and data by
  entrepreneurs, but about how to help them
\end{itemize}

\hypertarget{benefits-and-types-of-data}{%
\section*{Benefits and Types of Data}\label{benefits-and-types-of-data}}
\addcontentsline{toc}{section}{Benefits and Types of Data}

\markright{Benefits and Types of Data}

\begin{itemize}
\tightlist
\item
  ICT could help the business organisation to achieve a greater
  efficiency and lower cost in their operation.
\item
  Impact of data on business productivity, cost, revenue, and
  profitability (Gërguri-Rashiti et al., 2017).
\item
  Data and ICT to push innovation and business performance (Ramadani et
  al., 2016).
\item
  From a consumer perspective, ICT, especially the internet, has an
  impact on the way consumers purchase products and services.
\item
  The Internet more reliable and affordable for many customers. It is
  now easier for the consumer to find information and vendors of a
  product or service, not only from the local market, but also from the
  global market.
\item
  Having well developed and operationalised ICT is an important factor
  to build the competitive business advantage (Gërguri-Rashiti et al.,
  2017).
\end{itemize}

\hypertarget{index}{%
\subsection*{Index}\label{index}}
\addcontentsline{toc}{subsection}{Index}

\begin{enumerate}
\def\labelenumi{\arabic{enumi}.}
\tightlist
\item
  Context and Background
\end{enumerate}

\begin{itemize}
\tightlist
\item
  Entrepreneurship and tourism
\item
  Entrepreneurs, data and ICT
\item
  Characteristics of the entrepreneurial firm
\end{itemize}

\begin{enumerate}
\def\labelenumi{\arabic{enumi}.}
\setcounter{enumi}{1}
\tightlist
\item
  Understanding data
\end{enumerate}

\begin{itemize}
\tightlist
\item
  What is data?
\item
  Benefits of data
\item
  Types of Data
\item
  The Data life-cycle
\item
  Management of data in small enterprises
\end{itemize}

\begin{enumerate}
\def\labelenumi{\arabic{enumi}.}
\setcounter{enumi}{2}
\tightlist
\item
  The Entrepreneurial data-driven process
\end{enumerate}

\begin{itemize}
\tightlist
\item
  The classic vs data-driven entrepreneurial process
\end{itemize}

\begin{enumerate}
\def\labelenumi{\arabic{enumi}.}
\setcounter{enumi}{3}
\tightlist
\item
  Methods and Technologies
\end{enumerate}

\begin{itemize}
\tightlist
\item
  Deployment methods
\item
  Cloud and entrepreneurs
\item
  Low code
\item
  Big data for entrepreneurs
\end{itemize}

\begin{enumerate}
\def\labelenumi{\arabic{enumi}.}
\setcounter{enumi}{4}
\tightlist
\item
  Data considerations for Tourism Entrepreneurs
\end{enumerate}

\begin{itemize}
\tightlist
\item
  Ética de Datos y Emprendimiento
\item
  Finanzas de Datos para Emprendedores
\item
  Conocimiento del Cliente a través de los Datos
\item
  La Gobernanza de Datos en PYMEs Emprendedoras
\end{itemize}

\begin{enumerate}
\def\labelenumi{\arabic{enumi}.}
\setcounter{enumi}{5}
\tightlist
\item
  Starting-up a data-driven entrepreneur (Wrapping all up)
\end{enumerate}

\begin{itemize}
\tightlist
\item
  Stages
\item
  Leadership styles
\end{itemize}

\begin{enumerate}
\def\labelenumi{\arabic{enumi}.}
\setcounter{enumi}{6}
\tightlist
\item
  What's next
\end{enumerate}

\begin{itemize}
\tightlist
\item
  Role of AI
\item
  IoT
\item
  Big Data and analytics
\end{itemize}

\bookmarksetup{startatroot}

\hypertarget{turismo-datos-y-emprendimiento}{%
\chapter{Turismo, Datos y
Emprendimiento}\label{turismo-datos-y-emprendimiento}}

Autor: Cai Liang Zhou Cheng

Lorem ipsum dolor sit amet, consectetur adipiscing elit. Praesent
dapibus ut libero nec semper. In quam lorem, rutrum in nulla quis,
elementum volutpat odio. Phasellus felis nunc, semper eu sodales eu,
laoreet nec arcu. Sed ac magna quis sapien accumsan gravida. Etiam
tristique dui id elit egestas condimentum. Integer gravida fermentum
placerat. Ut hendrerit viverra ipsum, id vehicula magna tincidunt sed.
Nunc fermentum diam purus, non dignissim purus tincidunt vitae. Sed
tincidunt tortor vitae malesuada molestie. Sed aliquet, est a vulputate
aliquet, massa erat hendrerit arcu, in ultrices nulla nisl vel tortor.
Nam velit est, venenatis et tortor ac, rhoncus feugiat est. Nunc non
neque erat. Etiam eget ipsum fermentum risus lobortis consequat sit amet
sit amet dui. Maecenas auctor vehicula volutpat.

\hypertarget{titular-2}{%
\section{Titular}\label{titular-2}}

Etiam ultricies magna imperdiet nunc malesuada, ac lobortis sapien
rhoncus. Aenean eros lectus, accumsan vitae faucibus a, aliquam ac diam.
Maecenas finibus justo non nibh pharetra aliquam. Maecenas egestas, ex
vitae blandit convallis, leo mauris ornare nunc, porta fermentum tellus
est et urna. Aenean eu est et sapien laoreet placerat. Nunc turpis
ipsum, dapibus a ipsum at, tempus pretium lacus. In libero turpis,
tristique id orci a, auctor rhoncus risus. Praesent lacus nunc,
sollicitudin quis ipsum id, pulvinar venenatis mi.

\hypertarget{titular-3}{%
\subsection{Titular}\label{titular-3}}

Vivamus libero leo, accumsan non turpis vel, eleifend sodales mauris.
Donec pellentesque, tellus a lacinia vestibulum, nisl lacus dapibus
nibh, id tincidunt turpis erat non nisl. Donec ornare imperdiet metus,
eu sollicitudin risus elementum non. Aliquam metus eros, maximus
dignissim pellentesque nec, venenatis vitae purus. Vivamus laoreet mi
magna, ac malesuada tellus condimentum in. Integer nulla lectus, finibus
sit amet ex nec, finibus molestie diam. Pellentesque gravida ac erat sit
amet tempus.

\bookmarksetup{startatroot}

\hypertarget{muxe9todos-y-tecnologuxedas}{%
\chapter{Métodos y Tecnologías}\label{muxe9todos-y-tecnologuxedas}}

Autora: Ana Gabriela Echeverría Solís

\begin{enumerate}
\def\labelenumi{\arabic{enumi}.}
\tightlist
\item
  Intro al capítulo con los siguientes puntos: * Destacar la importancia
  de un buen método y selección de la tecnología adecuada al proyecto. *
  Orientadas a conocer los detalles y características del target. *
  Personalizar los productos, servicios, experiencias. 3. Método *
  Detectar oportunidades y amenazas: Análisis externo * Detectar
  fortalezas y debildiades: Análisis interno * Benchmarking: mejores
  prácticas * Tendencias y tecnologías emergentes * La innovación como
  ventaja competitiva 4. Tecnologías y soluciones Cómo elegir la mejor
  tecnología para solucionar los retos del proyecto * TIC
  -\textgreater{} Definición y cómo ayudan en la gestión. * Plataformas
  de reserva, ventas y gestión * Herramientas de márketing digigal * IoT
  / Wearables (QR dinámico, chips NFC, Beacon, Smart rooms) * CRM
  Turísticos * Web 3.0 * Entornos visuales /virtuales / Modelado 2D / 3D
  / Metaverso * Soluciones HW / SW * Proveedores de datos * Seguridad /
  Blockchain / Control de aforos * GIS * Cartelera inteligente / Tótems
  / Señalización turística * Chatbots / procesamiento del lenguaje
  natural * Desarrollo de Apps * Drones * Gestión eficiente de recursos:
  agua / energía / residuos / calidad del aire * Movilidad / Transporte
  * Plataformas de comunicación / Formación / Webinars * Sanidad *
  Sensorización * Sistemas de analítica de datos (Big data / BI)
\end{enumerate}

\bookmarksetup{startatroot}

\hypertarget{proceso-emprendedor}{%
\chapter{Proceso Emprendedor}\label{proceso-emprendedor}}

Autora: Karen Jualiana Fernández Castillo

Lorem ipsum dolor sit amet, consectetur adipiscing elit. Praesent
dapibus ut libero nec semper. In quam lorem, rutrum in nulla quis,
elementum volutpat odio. Phasellus felis nunc, semper eu sodales eu,
laoreet nec arcu. Sed ac magna quis sapien accumsan gravida. Etiam
tristique dui id elit egestas condimentum. Integer gravida fermentum
placerat. Ut hendrerit viverra ipsum, id vehicula magna tincidunt sed.
Nunc fermentum diam purus, non dignissim purus tincidunt vitae. Sed
tincidunt tortor vitae malesuada molestie. Sed aliquet, est a vulputate
aliquet, massa erat hendrerit arcu, in ultrices nulla nisl vel tortor.
Nam velit est, venenatis et tortor ac, rhoncus feugiat est. Nunc non
neque erat. Etiam eget ipsum fermentum risus lobortis consequat sit amet
sit amet dui. Maecenas auctor vehicula volutpat.

\hypertarget{titular-4}{%
\section{Titular}\label{titular-4}}

Etiam ultricies magna imperdiet nunc malesuada, ac lobortis sapien
rhoncus. Aenean eros lectus, accumsan vitae faucibus a, aliquam ac diam.
Maecenas finibus justo non nibh pharetra aliquam. Maecenas egestas, ex
vitae blandit convallis, leo mauris ornare nunc, porta fermentum tellus
est et urna. Aenean eu est et sapien laoreet placerat. Nunc turpis
ipsum, dapibus a ipsum at, tempus pretium lacus. In libero turpis,
tristique id orci a, auctor rhoncus risus. Praesent lacus nunc,
sollicitudin quis ipsum id, pulvinar venenatis mi.

\hypertarget{titular-5}{%
\subsection{Titular}\label{titular-5}}

Vivamus libero leo, accumsan non turpis vel, eleifend sodales mauris.
Donec pellentesque, tellus a lacinia vestibulum, nisl lacus dapibus
nibh, id tincidunt turpis erat non nisl. Donec ornare imperdiet metus,
eu sollicitudin risus elementum non. Aliquam metus eros, maximus
dignissim pellentesque nec, venenatis vitae purus. Vivamus laoreet mi
magna, ac malesuada tellus condimentum in. Integer nulla lectus, finibus
sit amet ex nec, finibus molestie diam. Pellentesque gravida ac erat sit
amet tempus.

\bookmarksetup{startatroot}

\hypertarget{pruxe1cticas-de-gestiuxf3n}{%
\chapter{Prácticas de Gestión}\label{pruxe1cticas-de-gestiuxf3n}}

Autora: Karen Paulina Salazar Núñez

Lorem ipsum dolor sit amet, consectetur adipiscing elit. Praesent
dapibus ut libero nec semper. In quam lorem, rutrum in nulla quis,
elementum volutpat odio. Phasellus felis nunc, semper eu sodales eu,
laoreet nec arcu. Sed ac magna quis sapien accumsan gravida. Etiam
tristique dui id elit egestas condimentum. Integer gravida fermentum
placerat. Ut hendrerit viverra ipsum, id vehicula magna tincidunt sed.
Nunc fermentum diam purus, non dignissim purus tincidunt vitae. Sed
tincidunt tortor vitae malesuada molestie. Sed aliquet, est a vulputate
aliquet, massa erat hendrerit arcu, in ultrices nulla nisl vel tortor.
Nam velit est, venenatis et tortor ac, rhoncus feugiat est. Nunc non
neque erat. Etiam eget ipsum fermentum risus lobortis consequat sit amet
sit amet dui. Maecenas auctor vehicula volutpat.

\hypertarget{titular-6}{%
\section{Titular}\label{titular-6}}

Etiam ultricies magna imperdiet nunc malesuada, ac lobortis sapien
rhoncus. Aenean eros lectus, accumsan vitae faucibus a, aliquam ac diam.
Maecenas finibus justo non nibh pharetra aliquam. Maecenas egestas, ex
vitae blandit convallis, leo mauris ornare nunc, porta fermentum tellus
est et urna. Aenean eu est et sapien laoreet placerat. Nunc turpis
ipsum, dapibus a ipsum at, tempus pretium lacus. In libero turpis,
tristique id orci a, auctor rhoncus risus. Praesent lacus nunc,
sollicitudin quis ipsum id, pulvinar venenatis mi.

\hypertarget{titular-7}{%
\subsection{Titular}\label{titular-7}}

Vivamus libero leo, accumsan non turpis vel, eleifend sodales mauris.
Donec pellentesque, tellus a lacinia vestibulum, nisl lacus dapibus
nibh, id tincidunt turpis erat non nisl. Donec ornare imperdiet metus,
eu sollicitudin risus elementum non. Aliquam metus eros, maximus
dignissim pellentesque nec, venenatis vitae purus. Vivamus laoreet mi
magna, ac malesuada tellus condimentum in. Integer nulla lectus, finibus
sit amet ex nec, finibus molestie diam. Pellentesque gravida ac erat sit
amet tempus.

\bookmarksetup{startatroot}

\hypertarget{oportunidades}{%
\chapter{Oportunidades}\label{oportunidades}}

Autora: Jenny María Checo Vargas

Temas de referencia: - a - b - c

Lorem ipsum dolor sit amet, consectetur adipiscing elit. Praesent
dapibus ut libero nec semper. In quam lorem, rutrum in nulla quis,
elementum volutpat odio. Phasellus felis nunc, semper eu sodales eu,
laoreet nec arcu. Sed ac magna quis sapien accumsan gravida. Etiam
tristique dui id elit egestas condimentum. Integer gravida fermentum
placerat. Ut hendrerit viverra ipsum, id vehicula magna tincidunt sed.
Nunc fermentum diam purus, non dignissim purus tincidunt vitae. Sed
tincidunt tortor vitae malesuada molestie. Sed aliquet, est a vulputate
aliquet, massa erat hendrerit arcu, in ultrices nulla nisl vel tortor.
Nam velit est, venenatis et tortor ac, rhoncus feugiat est. Nunc non
neque erat. Etiam eget ipsum fermentum risus lobortis consequat sit amet
sit amet dui. Maecenas auctor vehicula volutpat.

\hypertarget{titular-8}{%
\section{Titular}\label{titular-8}}

Etiam ultricies magna imperdiet nunc malesuada, ac lobortis sapien
rhoncus. Aenean eros lectus, accumsan vitae faucibus a, aliquam ac diam.
Maecenas finibus justo non nibh pharetra aliquam. Maecenas egestas, ex
vitae blandit convallis, leo mauris ornare nunc, porta fermentum tellus
est et urna. Aenean eu est et sapien laoreet placerat. Nunc turpis
ipsum, dapibus a ipsum at, tempus pretium lacus. In libero turpis,
tristique id orci a, auctor rhoncus risus. Praesent lacus nunc,
sollicitudin quis ipsum id, pulvinar venenatis mi.

\hypertarget{titular-9}{%
\subsection{Titular}\label{titular-9}}

Vivamus libero leo, accumsan non turpis vel, eleifend sodales mauris.
Donec pellentesque, tellus a lacinia vestibulum, nisl lacus dapibus
nibh, id tincidunt turpis erat non nisl. Donec ornare imperdiet metus,
eu sollicitudin risus elementum non. Aliquam metus eros, maximus
dignissim pellentesque nec, venenatis vitae purus. Vivamus laoreet mi
magna, ac malesuada tellus condimentum in. Integer nulla lectus, finibus
sit amet ex nec, finibus molestie diam. Pellentesque gravida ac erat sit
amet tempus.

\bookmarksetup{startatroot}

\hypertarget{uxe9tica-de-datos-y-emprendimiento}{%
\chapter{Ética de Datos y
Emprendimiento}\label{uxe9tica-de-datos-y-emprendimiento}}

Autora: Mensia Indira Otáñez de la Rosa

Lorem ipsum dolor sit amet, consectetur adipiscing elit. Praesent
dapibus ut libero nec semper. In quam lorem, rutrum in nulla quis,
elementum volutpat odio. Phasellus felis nunc, semper eu sodales eu,
laoreet nec arcu. Sed ac magna quis sapien accumsan gravida. Etiam
tristique dui id elit egestas condimentum. Integer gravida fermentum
placerat. Ut hendrerit viverra ipsum, id vehicula magna tincidunt sed.
Nunc fermentum diam purus, non dignissim purus tincidunt vitae. Sed
tincidunt tortor vitae malesuada molestie. Sed aliquet, est a vulputate
aliquet, massa erat hendrerit arcu, in ultrices nulla nisl vel tortor.
Nam velit est, venenatis et tortor ac, rhoncus feugiat est. Nunc non
neque erat. Etiam eget ipsum fermentum risus lobortis consequat sit amet
sit amet dui. Maecenas auctor vehicula volutpat.

\hypertarget{titular-10}{%
\section{Titular}\label{titular-10}}

Etiam ultricies magna imperdiet nunc malesuada, ac lobortis sapien
rhoncus. Aenean eros lectus, accumsan vitae faucibus a, aliquam ac diam.
Maecenas finibus justo non nibh pharetra aliquam. Maecenas egestas, ex
vitae blandit convallis, leo mauris ornare nunc, porta fermentum tellus
est et urna. Aenean eu est et sapien laoreet placerat. Nunc turpis
ipsum, dapibus a ipsum at, tempus pretium lacus. In libero turpis,
tristique id orci a, auctor rhoncus risus. Praesent lacus nunc,
sollicitudin quis ipsum id, pulvinar venenatis mi.

\hypertarget{titular-11}{%
\subsection{Titular}\label{titular-11}}

Vivamus libero leo, accumsan non turpis vel, eleifend sodales mauris.
Donec pellentesque, tellus a lacinia vestibulum, nisl lacus dapibus
nibh, id tincidunt turpis erat non nisl. Donec ornare imperdiet metus,
eu sollicitudin risus elementum non. Aliquam metus eros, maximus
dignissim pellentesque nec, venenatis vitae purus. Vivamus laoreet mi
magna, ac malesuada tellus condimentum in. Integer nulla lectus, finibus
sit amet ex nec, finibus molestie diam. Pellentesque gravida ac erat sit
amet tempus.

\bookmarksetup{startatroot}

\hypertarget{finanzas-de-datos-para-emprendedores}{%
\chapter{Finanzas de Datos para
Emprendedores}\label{finanzas-de-datos-para-emprendedores}}

Autora: Salomé Shirley Criollo Linaico

Lorem ipsum dolor sit amet, consectetur adipiscing elit. Praesent
dapibus ut libero nec semper. In quam lorem, rutrum in nulla quis,
elementum volutpat odio. Phasellus felis nunc, semper eu sodales eu,
laoreet nec arcu. Sed ac magna quis sapien accumsan gravida. Etiam
tristique dui id elit egestas condimentum. Integer gravida fermentum
placerat. Ut hendrerit viverra ipsum, id vehicula magna tincidunt sed.
Nunc fermentum diam purus, non dignissim purus tincidunt vitae. Sed
tincidunt tortor vitae malesuada molestie. Sed aliquet, est a vulputate
aliquet, massa erat hendrerit arcu, in ultrices nulla nisl vel tortor.
Nam velit est, venenatis et tortor ac, rhoncus feugiat est. Nunc non
neque erat. Etiam eget ipsum fermentum risus lobortis consequat sit amet
sit amet dui. Maecenas auctor vehicula volutpat.

\hypertarget{titular-12}{%
\section{Titular}\label{titular-12}}

Etiam ultricies magna imperdiet nunc malesuada, ac lobortis sapien
rhoncus. Aenean eros lectus, accumsan vitae faucibus a, aliquam ac diam.
Maecenas finibus justo non nibh pharetra aliquam. Maecenas egestas, ex
vitae blandit convallis, leo mauris ornare nunc, porta fermentum tellus
est et urna. Aenean eu est et sapien laoreet placerat. Nunc turpis
ipsum, dapibus a ipsum at, tempus pretium lacus. In libero turpis,
tristique id orci a, auctor rhoncus risus. Praesent lacus nunc,
sollicitudin quis ipsum id, pulvinar venenatis mi.

\hypertarget{titular-13}{%
\subsection{Titular}\label{titular-13}}

Vivamus libero leo, accumsan non turpis vel, eleifend sodales mauris.
Donec pellentesque, tellus a lacinia vestibulum, nisl lacus dapibus
nibh, id tincidunt turpis erat non nisl. Donec ornare imperdiet metus,
eu sollicitudin risus elementum non. Aliquam metus eros, maximus
dignissim pellentesque nec, venenatis vitae purus. Vivamus laoreet mi
magna, ac malesuada tellus condimentum in. Integer nulla lectus, finibus
sit amet ex nec, finibus molestie diam. Pellentesque gravida ac erat sit
amet tempus.

\bookmarksetup{startatroot}

\hypertarget{conocimiento-del-cliente-a-travuxe9s-de-los-datos}{%
\chapter{Conocimiento del Cliente a través de los
Datos}\label{conocimiento-del-cliente-a-travuxe9s-de-los-datos}}

Autora: Claudia Reina Vargas

\hypertarget{introducciuxf3n-1}{%
\section{Introducción}\label{introducciuxf3n-1}}

El sector turístico, se trata de un sector que, debido a la
globalización y la revolución digital del sistema entre otros factores a
destacar, se ha vuelto altamente competitivo. Por ello ha surgido la
necesidad de formular nuevas estrategias de marketing y análisis de
datos para la reinvención del sector en un destino en concreto. La alta
competencia y las facilidades que tiene el turista a la hora de poder
seleccionar el destino son claves para el crecimiento de este sector. Si
hablamos del caso de España, según los datos del Ministerio de Industria
Comercio y Turismo, en 2022 se superaron las cifras prepandemia por la
Covid-19, alcanzando los 71,6 millones de turistas internacionales que
realizaron un gasto de 87.061 millones de euros, provenientes de Reino
Unido, Alemania y Francia principalmente. En cuanto al gasto, los datos
arrojan un aumento del 10,5\% de media por turista frente a 2019 y las
estancias aumentaron a una media de 7,5 días de media. Observando estas
cifras, podemos ver que el turismo está sufriendo un gran auge, pero
¿hasta cuándo durará esto? El aumento del coste de vida ha sido notable
en muchas industrias, y puede, que ciertos modelos de turismo en países
estacionalizados no sea viable en un futuro. Ante este escenario
cambiante y de gran incertidumbre, es necesario que el sector del
turismo desarrolle continuamente nuevos proyectos para seguir siendo una
industria sólida, con una consolidada posición de liderazgo
internacional y con un potencial incansable (Consejo Español de Turismo,
2007). Muchas empresas, en sus inicios, estaban presentes únicamente en
un mercado concreto y en una tipología concreta de turismo, pero se
vieron obligadas a lanzarse al mercado internacional para poder
subsistir al crecimiento de otras entidades extranjeras. Pero el propio
consumidor ha cambiado también sus hábitos de compra y consumo de
servicios, teniendo mayor poder de decisión y negociación que las
propias empresas. A diario se reciben miles de estímulos que impulsan a
tomar unas u otras decisiones de compra. La investigación de mercados
constituye una parte fundamental para desarrollo de estrategias de
crecimiento y mejora, pero junto con un análisis correcto se deben
emplear otras herramientas y técnicas adecuadas. En este sentido, el uso
de big data, la minería de datos o técnicas de Business Intelligence
(BI) suponen una importante oportunidad para las empresas turísticas que
quieran aumentar sus ventas tanto para el público nacional como para el
internacional, ayudando con la innovación y el posicionamiento de los
destinos en diferentes mercados, conocimiento de la demanda y del
público objetivo entre otros factores.

\hypertarget{fuentes-de-datos-utilizadas-en-el-sector-turuxedstico}{%
\section{Fuentes de datos utilizadas en el sector
turístico}\label{fuentes-de-datos-utilizadas-en-el-sector-turuxedstico}}

El rápido crecimiento de las Tecnologías de la Información y la
Comunicación (TICs) ha generado grandes transformaciones sociales y
económicas a nivel mundial. En el sector turístico, se utilizan diversas
técnicas de análisis de datos para obtener información valiosa que
permita tomar decisiones estratégicas y mejorar la experiencia de los
clientes, así como para lograr el posicionamiento de un destino. Algunas
de estas técnicas incluyen la minería de datos, el aprendizaje
automático, la inteligencia artificial y la segmentación de clientes,
para cuando se tiene acceso a estas herramientas, pero tambiíen hay
otras como los datos proporcionados por webs oficiales de Goviernos y
Ministerios, las redes sociales (como Facebook, Twitter, Instagram o
Tripadvisor), diferentes estudios de mercado públicos en el sector que
deseen,etc., las cuales iremos desglosando para identificar sus ventajes
y aportes a esta industria. Estos sitemas ofrecen la información
necesaria a las empresas para tomar decisiones que permitan aumentar la
capacidad productiva, impulsar la competitividad y evaluar el retorno de
las acciones de fomento de la actividad turística. Por otra parte,
contribuye a incrementar las prestaciones a los visitantes adaptando la
oferta a las nuevas exigencias y necesidades. También sirve de ayuda a
los gestores de proyectos de destinos turísticos o para la correcta
elaboración de estrategias y planes de marketing o planes de acción.
Otro de sus usos es para averiguar el por qué un destino atrae más a un
determinado segmento y no por otro o los precios que pueden poner según
los departamentos de revenue. En este capítulo, se procederá a realizar
un desglose con mayor profundidad de cuáles son las herramientas que
ayudan en el sector, así como las aplicaciones que tienen para la
obtención de datos, sus diferencias y un caso práctico de una empresa
turística que recurra a estas herramientas para un análisis correcto. El
objetivo es poner en valor aquellas herramientas que con el paso del
tiempo se han podido desarrollar gracias a las TICs y como su uso mejora
y facilita las decisiones de las empresas, no solo del sector turístico,
sino de todos los implicados en la cadena de valor de un servicio, para
posteriormente centrarnos en como conseguir datos por fuera de estas
herramientas.

\hypertarget{smart-data}{%
\subsection{Smart Data}\label{smart-data}}

Para los emprendedores, la manera más sencilla de de otender datos sobre
el sector para emprendedores en la industria es mediante informes y
webs/apps y Smart Data. Este concepto se refiere a la utilización de
datos relevantes y significativos para obtener información y
conocimientos valiosos. Para los emprendedores en el sector turístico,
el smart data puede ser una herramienta poderosa para comprender mejor a
los clientes, tomar decisiones informadas y mejorar la eficiencia
operativa. Entre estas formas de búsqueda y aplicación podemos destacar,
por ejemplo, análisis de datos de clientes mediante informes
demográficos proporcionados de manera gratuita por diferentes entidades.
En el caso de España podemos destacar, el Instituto Nacional de
Estadística (INE), DATAESTUR (Selección de los datos más relevantes del
sector turístico), SEGITTUR (sociedad estatal española dedicada a la
gestión de la innovación y las tecnologías turísticas), EXCELTUR, y
muchas más. En sus webs se pueden encontrar análisis e informes del
sector, los cuales son de uso público y están a disposición de todo el
mundo, permitiendo así realizar análisis de referencias de viaje,
historial de reservas, demografía, comentarios y opiniones, para
comprender mejor a su audiencia. Por otro lado, se puede realizar un
monitoreo de redes sociales. Las redes sociales son una fuente valiosa
de smart data para los emprendedores turísticos. Al monitorear las
conversaciones en las redes sociales, las ubicaciones más usadas o en
tendencia, las fotografías de otros usuarios o de su competencia o
realizar encuestas los emprendedores pueden obtener información en
tiempo real sobre las opiniones de los clientes, las tendencias de
viaje, los destinos populares y las experiencias compartidas por los
viajeros. Esto puede ayudar a ajustar sus estrategias de marketing,
identificar oportunidades y responder rápidamente a las necesidades y
expectativas de los clientes. Por último, mediante el análisis de
opiniones web. Este hecho cada vez tiene más importancia tanto para
usuarios como para los empresarios, que buscan en ambos casos mostrar la
realidad que hay detrás de un negocio mediante opiniones públicas, a las
que tiene acceso todo el mundo, y las cuales se quedan guardadas en los
sitios webs, pudiendo ir acompañadas en algunos casos de vídeos o
fotografías. Al obtener esta información de su negocio y de sus
competidores, los emprendedores pueden identificar áreas de mejora,
tomar medidas correctivas y mantener la satisfacción del cliente en un
nivel alto. Pero estas no son las únicas tecnologías o pasos que puede
seguir un emprendedor para analizar al turista o el sector, existen como
ya hemos mencionado tecnologías más potentes, las cuales ahorran tiempo
y generan informes de toda la información, pudiendo conseguir una
adaptación de estrategias, segmentaciones y ajuste de precios entre
otros factores más rápida.

\hypertarget{mineruxeda-de-datos}{%
\subsection{Minería de Datos}\label{mineruxeda-de-datos}}

En los últimos años, debido al desarrollo exponencial de la tecnología,
se genera cada vez más información, la cual es más difícil de analizar,
pero a su vez el acceso a ella es más sencillo. Estas grandes cantidades
de datos disponibles online o almacenados en bases de datos, son una
oportunidad a la hora de obtener nueva información que es potencialmente
importante pero que aún no ha sido descubierta. Se hace referencia al
concepto de ``Big Data'' cuando esta información, debido a su volumen y
complejidad, no puede ser procesada o analizada utilizando procesos o
herramientas tradicionales. Berman (2013) propone que se defina Big Data
a partir de las tres V's: volumen, variedad y velocidad. Cuando habla de
volumen se refiere a contar con una gran cantidad de datos. La variedad
está dada por las diferentes formas y formatos que tienen esos datos, ya
sea en bases de datos tradicionales, imágenes, documentos y registros
complejos. Por último, la velocidad hace alusión a que el contenido de
los datos cambia constantemente, ya sea a través de la generación de
nuevos datos, de la absorción de colecciones de datos complementarias o
de la introducción de datos previamente archivados. El análisis de esta
información suele realizarse por personas expertas en la materia, pero
este proceso tiene altos costes monetarios y requiere de mucho tiempo.
Es aquí cuando entra el juego la minería de datos o data mining, ya que
esta tecnología se sirve de estos sistemas utilizando algoritmos
específicos para extraer patrones desde los datos, permitiendo encontrar
patrones repetitivos que se encuentran escondidos en esos datos. Para
ello combina la estadística, la inteligencia artificial, el aprendizaje
automático (machine learning) y la tecnología de bases de datos.
\includegraphics{index_files/mediabag/ebf9853f-4644-4ba6-a.pdf} Una vez
extraída la información de la minería de datos, se puede usar para
predicciones de comportamientos, demandas y ventajas competitivas.
\includegraphics{index_files/mediabag/539771e1-4a7c-4164-8.pdf} En
cuanto a su aplicación en el sector turístico podemos destacar
diferentes posibilidades: Por un lado se utiliza para el análisis de
datos de reservas de hoteles para identificar los patrones de reserva,
como la antelación con la que los clientes realizan sus reservas, las
temporadas de mayor demanda, los tipos de habitación más populares, etc.
Al analizar los datos de reservas de hoteles, se puede determinar la
antelación con la que los clientes realizan sus reservas. Esto permite
identificar tendencias y patrones en cuanto al período de tiempo entre
la fecha de reserva y la fecha de llegada. Por ejemplo, se puede
descubrir que los viajeros de negocios tienden a reservar con menos
antelación que los turistas de ocio. También permite identificar las
temporadas de mayor demanda en la industria hotelera. Al analizar las
fluctuaciones en las reservas a lo largo del tiempo, se pueden
identificar patrones estacionales y determinar cuándo hay una mayor
demanda de habitaciones. Esta información es esencial para establecer
estrategias de precios y planificación de la capacidad. Otro beneficio
sería la duración de la estancia o el canal de la reserva. En cuanto a
la duración de la estancia, el data mining sirve como herramienta de
elaboración de informes de conducta, diferenciación de tipologías de
clientes y predicciones a la hora de establecer precios. Además, ayuda a
recordar eventos que han sucedido en lugares concretos y por los cuales
las estancias se vieron afectadas, pudiendo alargarse o acortarse, y
sirve como base de datos para años futuros. Por otro lado, analizar los
canales de reserva puede ayudar en el proceso de establecimiento de una
estrategia, en la cual e puede ver que canal es el más usado, ya sea
online mediante OTAs como Booking, Expedia, Hotelbeds, etc., recurriendo
a los servicios de reserva directos por parte del alojamiento/servicio
como su propia web o mediante un CRS concreto (programa de central de
reservas u oficinas donde gestionan las mismas, ya sea mediante email o
teléfono).

\hypertarget{inteligencia-artificial-ia}{%
\subsection{Inteligencia Artificial
(IA)}\label{inteligencia-artificial-ia}}

La inteligencia artificial (IA) es un campo de estudio y desarrollo que
se enfoca en crear sistemas capaces de realizar tareas que normalmente
requerirían de la inteligencia humana. La idea principal detrás de la IA
es emular ciertos aspectos de la capacidad humana para aprender,
razonar, comprender, planificar, tomar decisiones y resolver problemas.
Se basa en la idea de que las máquinas pueden procesar información de
manera similar a los seres humanos, y utilizarla para tomar decisiones o
realizar acciones de manera autónoma. Los sistemas de IA pueden aprender
de datos, adaptarse a nuevas situaciones y mejorar su rendimiento a lo
largo del tiempo, sin necesidad de ser programados de forma explícita
para cada tarea específica. Existen diferentes enfoques y técnicas
dentro del campo de la IA. Uno de los enfoques más utilizados es el
aprendizaje automático (machine learning), que se basa en algoritmos y
modelos matemáticos que permiten a las máquinas aprender a través de la
experiencia y los datos, del cual hablaremos más adelante. También
existen otros enfoques dentro de la IA, los cuales incluyen el
procesamiento de lenguaje natural (NLP, por sus siglas en inglés), que
se enfoca en la comprensión y generación de lenguaje humano, la visión
por ordenador, que permite a las máquinas analizar y comprender imágenes
y videos, la robótica, que busca desarrollar robots capaces de
interactuar con el entorno de manera inteligente y la planificación y
toma de decisiones, que involucra la capacidad de los sistemas de IA
para tomar decisiones óptimas en situaciones complejas. La IA se utiliza
en una amplia variedad de industrias y campos, incluyendo el sector
turístico. En el turismo, la IA ha encontrado aplicaciones en áreas como
la atención al cliente, la personalización de ofertas, la gestión de
destinos, el análisis de datos de viajeros, la mejora de la seguridad y
la optimización de precios. A través de chatbots y asistentes virtuales,
las empresas turísticas pueden brindar atención al cliente las 24 horas
del día, los 7 días de la semana, respondiendo preguntas y ofreciendo
recomendaciones personalizadas. Mediante el análisis de datos de los
clientes, la IA permite a las empresas comprender mejor las preferencias
y necesidades de los viajeros, ofreciendo ofertas y promociones más
relevantes. En la gestión de destinos, la IA puede analizar datos de
geolocalización y comportamiento de los turistas para mejorar la
planificación de rutas, identificar áreas de interés y optimizar la
experiencia del viajero. Además, la IA también se utiliza en la
detección de fraudes y la mejora de la seguridad en el sector turístico,
ayudando a identificar actividades sospechosas y proteger tanto a los
turistas como a las empresas. Se utiliza también para analizar
comentarios y reseñas de clientes en plataformas en línea. Al aplicar
técnicas de procesamiento de lenguaje natural, la IA puede identificar
el tono emocional de los comentarios (positivo, negativo o neutral) y
extraer información útil sobre las experiencias de los clientes. Esto
ayuda a los proveedores de servicios turísticos a comprender mejor las
necesidades y expectativas de sus clientes. Por otro lado, debido al
auge de los dispositivos móviles y las aplicaciones de viaje, que
generan grandes cantidades de datos de geolocalización, gracias a la IA
se puede analizar estos datos para comprender los patrones de movilidad
de los turistas, identificar lugares de interés populares y pronosticar
la congestión de tráfico en ciertas áreas. Esto ayuda a los proveedores
de servicios turísticos a mejorar la planificación de rutas y optimizar
la gestión de destinos. Por último, en cuanto al diseño de estrategias
de marketing, la IA puede identificar patrones y segmentar a los
clientes en grupos con características similares, esto se hace mediante
el procesamiento de grandes cantidades de datos de clientes, como
preferencias de viaje, comportamiento de compra, historial de reservas,
interacciones en redes sociales, entre otros, lo que ayuda a la creación
de estas campañas por segmentos para que tengan mayor efectividad. El
objetivo principal de la segmentación de clientes es comprender mejor a
los diferentes grupos de clientes y adaptar las estrategias de marketing
y servicios para satisfacer sus necesidades de manera más efectiva. La
segmentación de clientes se basa en la recopilación y análisis de datos
relevantes, como datos demográficos, preferencias de viaje,
comportamiento de compra, historial de reservas, entre otros. Estos
datos se utilizan para identificar patrones y tendencias en el
comportamiento de los clientes, lo que permite agruparlos en segmentos
con características similares.

La inteligencia artificial (IA) desempeña un papel importante en la
segmentación de clientes, ya que puede procesar grandes volúmenes de
datos y encontrar patrones y relaciones complejas de manera más
eficiente que los métodos tradicionales. Los algoritmos de IA utilizados
en la segmentación de clientes pueden aplicar técnicas de aprendizaje
automático para identificar relaciones no lineales y comprender las
preferencias y comportamientos de los clientes de manera más precisa.
Tal y como habíamos mencionado antes, La inteligencia artificial y el
aprendizaje automático (machine learning) están estrechamente
relacionados y se considera que el aprendizaje automático es una subrama
de la IA. La inteligencia artificial, por su lado, se refiere al campo
de estudio y desarrollo de sistemas que pueden realizar tareas que
normalmente requerirían de la inteligencia humana buscando emular
ciertos aspectos de la inteligencia humana, mientras que el aprendizaje
automático, por otro lado, es una técnica dentro de la IA que se basa en
algoritmos y modelos matemáticos para permitir que las máquinas aprendan
de datos y experiencias. El objetivo del aprendizaje automático es
desarrollar modelos y algoritmos que puedan aprender a reconocer
patrones, hacer predicciones, tomar decisiones y resolver problemas a
través de la experiencia y los datos En otras palabras, el aprendizaje
automático es una herramienta utilizada en la implementación de la
inteligencia artificial.

\hypertarget{aprendizaje-automuxe1tico}{%
\subsection{Aprendizaje automático}\label{aprendizaje-automuxe1tico}}

El aprendizaje automático, también conocido como machine learning, es
una técnica ampliamente utilizada en el sector turístico para el
análisis de datos y la toma de decisiones informadas, siendo una
subdivisión de la Inteligencia Artificial que se basa en algoritmos para
analizar conjuntos de datos masivos. Los datos se convierten en
información, conocimiento y por último en una herramienta de sabiduría
para realizar acciones con una base sólida. Estos algoritmos permiten a
las máquinas aprender y mejorar automáticamente a partir de los datos
sin ser programadas explícitamente. El objetivo del Machine Learning es
desarrollar sistemas que puedan analizar y comprender grandes cantidades
de datos, identificar patrones, hacer predicciones o tomar decisiones
basadas en esos patrones identificados. En lugar de seguir reglas y
algoritmos predefinidos, el Machine Learning se basa en la capacidad de
las máquinas para aprender de los datos y ajustar sus acciones o
predicciones en función de esa experiencia. El proceso de Machine
Learning generalmente se divide en tres etapas principales: -
Entrenamiento: Durante esta etapa, se proporciona a los algoritmos de
Machine Learning un conjunto de datos de entrenamiento que contiene
ejemplos y respuestas esperadas. El algoritmo utiliza estos datos para
aprender patrones y crear un modelo que pueda hacer predicciones o tomar
decisiones. - Prueba y validación: Una vez que se ha entrenado el
modelo, se prueba su rendimiento utilizando un conjunto de datos de
prueba independiente. Este paso es importante para verificar si el
modelo puede hacer predicciones precisas o tomar decisiones acertadas. -
Implementación y mejora: Después de que el modelo ha sido probado y
validado, se implementa en un entorno de producción para su uso real. A
medida que se recopilan nuevos datos, el modelo puede seguir mejorando y
ajustándose mediante técnicas como el entrenamiento continuo o el
aprendizaje en línea. Existen diferentes enfoques y técnicas que se
utilizan para desarrollar modelos y algoritmos: - Aprendizaje
supervisado: En este enfoque, el modelo se entrena utilizando un
conjunto de datos etiquetados, es decir, datos que contienen ejemplos y
sus respuestas correspondientes. El objetivo es que el modelo aprenda a
hacer predicciones o clasificar nuevos datos basándose en los patrones
identificados en el conjunto de entrenamiento. - Aprendizaje no
supervisado: En este caso, el modelo se entrena utilizando un conjunto
de datos no etiquetados, es decir, datos sin respuestas previas. El
objetivo es que el modelo descubra patrones y estructuras ocultas en los
datos, agrupándolos o realizando otras tareas de análisis sin necesidad
de respuestas predefinidas. - Aprendizaje por refuerzo: En este enfoque,
el modelo aprende a través de la interacción con un entorno y la
retroalimentación recibida. El modelo toma acciones en un entorno y
recibe recompensas o penalizaciones según su desempeño. El objetivo es
que el modelo aprenda a tomar decisiones óptimas para maximizar la
recompensa a largo plazo. - Aprendizaje profundo (Deep Learning): Es una
técnica que se basa en redes neuronales artificiales profundas para
aprender y extraer características complejas de los datos. Las redes
neuronales profundas están compuestas por múltiples capas y pueden
aprender representaciones de alto nivel de los datos, lo que las hace
especialmente útiles para tareas como el reconocimiento de imágenes, el
procesamiento del lenguaje natural y la traducción automática Si
hablamos de la aplicación de esta metodología para el sector turístico
podemos destacar diferentes usos. Una de las más conocidas sería la
personalización de recomendaciones ya que los algoritmos de aprendizaje
automático se utilizan para analizar los datos de los usuarios, como sus
preferencias, historial de reservas y comportamiento de navegación, con
el fin de ofrecer recomendaciones personalizadas. Estos algoritmos
pueden identificar patrones y tendencias para ofrecer opciones de viaje
relevantes y adaptadas a las preferencias individuales de los usuarios,
algo que el usuario cada vez nota más, ya que cuando busca un
alojamiento, puede que le aparezcan primero los resultados de otras
cadenas en las que se ha alojado o compañías aéreas con las que ha
volado. La demanda futura también se puede predecir con esta técnica, ya
que analiza grandes volúmenes de datos históricos, así como datos en
tiempo real, para predecir la demanda futura y ajustar los precios de
manera óptima. Esto ayuda a las empresas turísticas a establecer precios
competitivos, maximizar los ingresos y gestionar la capacidad de manera
eficiente. Uno de los usos más recientes de esta metodología es la
aplicación de algoritmos para analizar comentarios de los usuarios en
redes sociales, sitios web de reseñas y otras fuentes de
retroalimentación. Esto permite a las empresas turísticas comprender las
opiniones y los sentimientos de los clientes, identificar tendencias y
áreas de mejora, y responder de manera oportuna a los comentarios, pero
también ayuda con la detección de fraudes en reservas, identificando
patrones y comportamientos sospechosos en las transacciones. Esto ayuda
a prevenir actividades fraudulentas y proteger tanto a las empresas
turísticas como a los clientes. Además, se utiliza en sistemas de
seguridad para detectar amenazas y riesgos potenciales. En general, el
aprendizaje automático ofrece beneficios significativos en el sector
turístico, permitiendo a las empresas tomar decisiones más eficientes y
personalizadas, optimizar la experiencia del cliente, gestionar mejor
los recursos y adaptarse rápidamente a las tendencias y cambios del
mercado.

\hypertarget{big-data}{%
\subsection{Big Data}\label{big-data}}

De acuerdo con M. Chen et al (2014) y según se cita en Lukosius y Hyman
(2018), el big data son ``conjuntos de datos que no podrían percibirse,
adquirirse, gestionarse y procesarse por medio de TICs tradicionales y
herramientas de software/hardware dentro de unos tiempos razonables''.
Para el mundo empresarial, el Big Data ha cobrado más y más importancia,
el cual les da información vital sobre los consumidores, como sus
gustos, sentimientos, opiniones, preferencias, comportamientos
recurrentes, etc., extrayendo conclusiones de cantidades ingentes de
información para su posterior análisis. Estos conjuntos de datos son
generados a gran velocidad y en variedad de formatos, como datos
estructurados (por ejemplo, bases de datos), datos no estructurados
(como texto, imágenes o videos) y datos semiestructurados (como archivos
XML o JSON). En el sector turístico, el Big Data juega un papel
fundamental en la obtención y análisis de datos de los clientes. A
medida que los avances tecnológicos y la digitalización han transformado
la forma en que los turistas interactúan con las empresas y los destinos
turísticos, se ha generado una gran cantidad de datos relacionados con
los viajes y las experiencias de los clientes. En el caso del Big Data,
presenta unas características más particulares que las del resto de
herramientas. Por un lado, podemos destacar el volumen, y es que los
conjuntos de datos pueden abarcar desde terabytes hasta petabytes o
incluso exabytes de información. Esta cantidad de datos proviene de
diversas fuentes, como redes sociales, sensores, transacciones,
registros, entre otros. Otro aspecto es la velocidad, dado que los datos
son generados a y actualizados a altas velocidades, y esque en este tipo
de herramienta se requiere una respuesta rápida para aprovechar las
oportunidades o solucionar problemas. Como ya hemos mencionado, el Big
Data abarca diferentes tipos de datos y esto hace que se planteen
desafíos adicionales para su almacenamiento y procesamiento. Por último,
se debe tener en cuenta que los datos pueden ser incompletos,
inconsistentes o incorrectos. Debido a esto, la veracidad de los datos
puede ser un desafío, ya que es necesario validar y limpiar los datos
para garantizar su calidad y fiabilidad. Por tanto podemos decir que las
características principales del Big Data pueden agruparse dentro de la
regla de las 4 ``V''.

\hypertarget{aplicaciuxf3n-de-las-tecnologuxedas-al-emprendedor}{%
\subsection{Aplicación de las tecnologías al
emprendedor}\label{aplicaciuxf3n-de-las-tecnologuxedas-al-emprendedor}}

La aplicación de smart data coo tecnología junto con IA, Bid Data y el
reto de las tecnologías a gran escala implica utilizar técnicas y
herramientas avanzadas de análisis de datos para obtener información
relevante y tomar decisiones inteligentes en el contexto de un negocio o
proyecto emprendedor. Antes de empezar, es fundamental tener claridad
sobre los objetivos que se desealograr con el análisis de datos. ¿Se
busca identificar oportunidades de mercado? ¿Optimizar estrategias de
marketing? ¿Mejorar la eficiencia operativa? Establecer objetivos claros
ayudará a enfocar los esfuerzos durante los comienzos de nuevos
proyectos. Una vez se han establecido cuales son los objetivos a lograr
con los medios e información que tenemos se debe tener claro cuales son
las fuentes de datos que se van a usar, es decir, las que son más
relevantes para el negocio. Ya seleccionadas las fuentes hay que
asegurarse que los datos sean claros, precisos y estén actualizados.
Como hemos visto en los anteriores apartados, al tratarse de grandes
cantidades de información para las nuevas tecnologías y sobre todo, las
que son de libre uso como las redes sociales pueden tener datos falsos,
debiendo tener cuidado de no medir con esta información. Estos datos
deben ser recopilados, clasificados y almacenados con herramientas de
gestión de bases de datos o plataformas de almacenamiento en la nube
para facilitar este proceso debiendo asegurarse del cumplimiento de las
diferentes normativas de GDPR, pero a su vez los datos recopilados
pueden contener errores, valores atípicos o información irrelevante. Es
necesario realizar una limpieza y preprocesamiento de los datos para
eliminar inconsistencias y asegurar su calidad. Esto implica tareas como
eliminar registros duplicados, corregir errores y completar datos
faltantes. Es aquí cuando entra en juego la aplicación de técnicas de
análisis como el machine learning o la minería de datos y análisis
estadístico para extraer información valiosa de los datos. Estas
técnicas permitirán identificar patrones, tendencias, correlaciones y
otros conocimientos relevantes para el negocio. Junto con el análisis se
debe hacer una interpretación de los datos mediante gráficos y tablas.
Una vez obtenidos los resultados del análisis, es importante
interpretarlos en el contexto de cada negocio, evaluando los hallazgos
obtenidos y utilizándolos como base para tomar decisiones informadas. La
aplicación de Smart Data es un proceso continuo. A medida que se obtenga
más datos y se realicen nuevos análisis, es importante mejorar los
enfoques, aprendiendo de los resultados obtenidos y ajustando las
estrategias para optimizar la toma de decisiones.

\hypertarget{beneficios-y-limitaciones-del-anuxe1lisis-de-datos-en-el-sector-turuxedstico}{%
\subsection{Beneficios y limitaciones del análisis de datos en el sector
turístico}\label{beneficios-y-limitaciones-del-anuxe1lisis-de-datos-en-el-sector-turuxedstico}}

Como podemos ver, estas herramientas han supuesto un gran avance en el
sector turístico en general, y su aplicación permite potenciar a las
empresas en su crecimiento y optimización, pero al igual que todo, estas
herramientas y métodos tienen su parte positiva y sus beneficios y sus
limitaciones. Por un lado, destacamos como beneficios la gran ayuda que
son a la hora de la toma de decisiones informadas. El análisis de datos
permite a las empresas turísticas tomar decisiones basadas en evidencia
y datos reales. Puede proporcionar información valiosa sobre patrones de
reserva, preferencias de los clientes, tendencias del mercado y
comportamiento del consumidor, lo que permite a las empresas adaptar sus
estrategias y crear campañas con base de conocimiento sólida. La
personalización de servicios es posible gracias a estos métodos. El
análisis de datos permite a las empresas turísticas comprender mejor a
sus clientes y ofrecer servicios más personalizados. Al analizar datos
demográficos, preferencias y comportamiento de reserva, las empresas
pueden adaptar sus ofertas, promociones y recomendaciones para
satisfacer las necesidades individuales de los clientes y mejorar su
experiencia de viaje. También son de gran ayuda en cuanto a la
optimización de precios y ofertas se refiere, permitiendo a las empresas
turísticas ajustar los precios y las ofertas de manera más precisa. Al
analizar datos de mercado, demanda, competencia y otros factores
relevantes, las empresas pueden establecer precios competitivos y
estrategias de precios dinámicos que maximicen los ingresos y la
ocupación de los alojamientos. Por último, debemos destacar la mejora de
la gestión operativa, ayudando a mejorar la eficiencia y la gestión
operativa. Al analizar datos sobre la ocupación de los alojamientos, la
disponibilidad de recursos, la gestión de inventario y otros factores
operativos, las empresas pueden optimizar la asignación de recursos,
reducir costos y mejorar la satisfacción del cliente. También aparecen
limitaciones e inconvenientes que pueden causar dificultades a las
empresas. Por un lado, la calidad de los datos debe ser buena, y en el
sector turístico no siempre ocurre, debido a que pueden estar dispersos
en diferentes plataformas, pueden ser datos falsos (como las opiniones
en Google o Tripadvisor que no tienen control) o pueden contener
errores, lo que dificulta el uso efectivo de los mismos en análisis. Un
aspecto muy importante que se debe tener en cuenta es la privacidad y la
protección de datos de la información de cada cliente. El sector
turístico maneja datos personales sensibles de los clientes, como
información de identificación, preferencias de viaje y detalles de pago.
Es fundamental garantizar la privacidad y la protección de estos datos.
El análisis de datos debe cumplir con las regulaciones de privacidad y
seguridad de datos, como el Reglamento General de Protección de Datos
(GDPR) en la Unión Europea, lo que puede imponer restricciones al acceso
y uso de los datos, pero también debe tener en cuenta las leyes de cada
país, ya que son diferentes. Se requiere además una adecuada
interpretación y contextualización de los resultados. Los datos por sí
solos no proporcionan respuestas definitivas, sino que requieren una
comprensión profunda del dominio turístico y la capacidad de analizar
los resultados en el contexto adecuado. La interpretación errónea de los
datos puede llevar a decisiones equivocadas o conclusiones incorrectas.
No debemos olvidarnos de que el sector turístico es altamente dinámico y
está sujeto a cambios rápidos en las preferencias de los clientes, las
tendencias de viaje y las condiciones del mercado, por lo que cuando se
aplican estas herramientas de análisis se debe tener una actualización
regular de los modelos y algoritmos utilizados para obtener información
relevante y actualizada.

\hypertarget{reflexiuxf3n-sobre-los-desafuxedos-y-las-oportunidades-futuras-en-este-uxe1mbito}{%
\subsection{Reflexión sobre los desafíos y las oportunidades futuras en
este
ámbito}\label{reflexiuxf3n-sobre-los-desafuxedos-y-las-oportunidades-futuras-en-este-uxe1mbito}}

Algo que ha quedado claro es que el turismo es una actividad
interdependiente de otras industrias por lo que, al pausarse esta,
sectores como la movilidad, la logística, la salud, la agricultura, la
automoción o la alimentación, entre otros, se vieron también afectados.
Viendo esto, y si pensamos, por ejemplo, en el caso de España, el PIB
cerró el año 2022 con 159.490 millones de euros, cifra que es un 1,4\%
superior a la del año 2019, según el informe trimestral de Perspectivas
turísticas de la Alianza para la Excelencia Turística (Exceltur). El
sector crece cada vez más y aporta más valor, pero las fuentes de datos
se vuelven más amplias y complicadas de manjar en términos de análisis.
De ahí, que numerosos gobiernos, entre los que destacamos el caso de
España, se han propuesto crear espacios ``inteligentes'' en los cuales
se aporte seguridad a la hora de las empresas compartan sus datos con
otras del sector o en general, por lo que promueven el intercambio y
combinación de datos. La generación de estos espacios de compartición y
explotación de datos supondrá grandes ventajas para el sector, ya que se
facilitará la creación de ofertas, productos y servicios más
personalizados que proporcionen una experiencia mejorada y adaptada a
las necesidades de los clientes, mejorando así la capacidad de atraer
turistas. En términos generales generales y como desafíos podemos
destacar la mejora del tamaño y la calidad de los datos, siendo este uno
de los mayores desafíos ya que la gestión de grandes volúmenes de datos
sigue siendo una complicación y por tanto garantizar su calidad puede
complicarse. La industria turística genera una gran cantidad de
información, desde reservas y transacciones hasta interacciones en redes
sociales y opiniones de los clientes. La recopilación, integración y
limpieza de estos datos de manera eficiente y precisa puede ser un
desafío técnico. En cuanto al cumplimiento del GDPR pueden surgir
complicaciones debido a las diferentes normativas que existen en
diferentes países y marcos legislativos, por lo que deben garantizar que
los datos de los clientes estén protegidos contra accesos no autorizados
o brechas de seguridad. La interpretación correcta de los resultados y
la capacidad de traducirlos en acciones concretas pueden ser
desafiantes. Se requiere una combinación de experiencia en el sector
turístico y habilidades analíticas para aprovechar al máximo el análisis
de datos. A pesar de estos desafíos, también existen oportunidades que
pueden conseguir en el futuro gracias a los desarrollos tecnológicos,
como pueden ser la personalización y experiencia del cliente al
comprender las preferencias individuales de los clientes, las empresas
pueden ofrecer recomendaciones y ofertas personalizadas, lo que mejora
la satisfacción del cliente y fomenta la fidelidad. En cuanto a la
predicción y anticipación de la demanda se verán beneficiados utilizando
técnicas avanzadas de análisis y modelos predictivos, las empresas
turísticas pueden anticipar la demanda futura y ajustar su oferta en
consecuencia. Por otro lado, el uso más reciente pero con mayor auge
será el de las redes sociales, lo que permitirá una lectura correcta de
los comentarios y opiniones además del análisis de las ubicaciones de
los extranjeros y nacionales en un destino puntual para ver los
comportamientos.

\hypertarget{referencias}{%
\subsection{Referencias}\label{referencias}}

Las fuentes estadísticas y su rentabilidad en el sector turístico.
(s/f). Cehat.com. Recuperado el 8 de junio de 2023, de
https://cehat.com/las-fuentes-estadisticas-y-su-rentabilidad-en-el-sector-turistico/
EP. (2023, enero 17). El PIB turístico cerró 2022 recuperando el nivel
prepandemia. Ediciones EL PAÍS S.L.
https://cincodias.elpais.com/cincodias/2023/01/17/companias/1673960751\_828579.html
Es, D. G. (2023, febrero 7). Radiografía del dataspace nacional de
Turismo: retos y oportunidades para el sector turístico. datos.gob.es.
https://datos.gob.es/es/blog/radiografia-del-dataspace-nacional-de-turismo-retos-y-oportunidades-para-el-sector-turistico
El Sistema de Inteligencia Turística ayuda a los destinos a conocer las
necesidades de los turistas. (2016, junio 2). SEGITTUR.
https://www.segittur.es/sala-de-prensa/notas-de-prensa/el-sistema-de-inteligencia-turistica-ayuda-a-los-destinos-a-conocer-las-necesidades-de-los-turistas/
Sabadell, B. (s/f). El Blog de Banco Sabadell. El Blog de Banco
Sabadell. Recuperado el 8 de junio de 2023, de
https://blog.bancsabadell.com/2018/04/las-cuatro-v-del-big-data-volumen-variedad-velocidad-veracidad-que-es-el-big-data-para-que-sirve-con-sabadell-campus.html
Hosteltur. (2023, marzo 24). Airbnb supera las cifras prepandemia a lo
grande. Hosteltur.
https://www.hosteltur.com/156656\_airbnb-supera-las-cifras-prepandemia-a-lo-grande.html
Hosteltur. (s/f). Las 6 claves del uso del Big Data en el turismo.
Hosteltur: Toda la información de turismo. Recuperado el 8 de junio de
2023, de
https://www.hosteltur.com/comunidad/005340\_las-6-claves-del-uso-del-big-data-en-el-turismo.html
Juan, C. (2023). Las oportunidades de negocio que ofrece el Big Data en
el sector turístico. Thinking for Innovation.
https://www.iebschool.com/blog/big-data-en-el-sector-turistico-big-data/
Braintrust-Ekm, C. (2019, septiembre 19). Why should the tourism company
know its customers? BrainTrust CS.
https://www.braintrust-cs.com/en/empresa-turistica-conocer-clientes/
López, R. G. (2022, junio 30). Tecnologías que transforman el turismo.
Marketing Turístico Digital - Las nuevas tendencias del Marketing
Digital incorporadas al sector turístico.
https://marketingturisticodigital.com/2022/06/30/tecnologia-y-turismo/

\bookmarksetup{startatroot}

\hypertarget{la-gobernanza-de-datos-en-pymes-emprendedoras}{%
\chapter{La Gobernanza de Datos en PYMEs
Emprendedoras}\label{la-gobernanza-de-datos-en-pymes-emprendedoras}}

Autora: Bielka Yuvnny Ulloa Reynoso

Lorem ipsum dolor sit amet, consectetur adipiscing elit. Praesent
dapibus ut libero nec semper. In quam lorem, rutrum in nulla quis,
elementum volutpat odio. Phasellus felis nunc, semper eu sodales eu,
laoreet nec arcu. Sed ac magna quis sapien accumsan gravida. Etiam
tristique dui id elit egestas condimentum. Integer gravida fermentum
placerat. Ut hendrerit viverra ipsum, id vehicula magna tincidunt sed.
Nunc fermentum diam purus, non dignissim purus tincidunt vitae. Sed
tincidunt tortor vitae malesuada molestie. Sed aliquet, est a vulputate
aliquet, massa erat hendrerit arcu, in ultrices nulla nisl vel tortor.
Nam velit est, venenatis et tortor ac, rhoncus feugiat est. Nunc non
neque erat. Etiam eget ipsum fermentum risus lobortis consequat sit amet
sit amet dui. Maecenas auctor vehicula volutpat.

\hypertarget{titular-14}{%
\section{Titular}\label{titular-14}}

Etiam ultricies magna imperdiet nunc malesuada, ac lobortis sapien
rhoncus. Aenean eros lectus, accumsan vitae faucibus a, aliquam ac diam.
Maecenas finibus justo non nibh pharetra aliquam. Maecenas egestas, ex
vitae blandit convallis, leo mauris ornare nunc, porta fermentum tellus
est et urna. Aenean eu est et sapien laoreet placerat. Nunc turpis
ipsum, dapibus a ipsum at, tempus pretium lacus. In libero turpis,
tristique id orci a, auctor rhoncus risus. Praesent lacus nunc,
sollicitudin quis ipsum id, pulvinar venenatis mi.

\hypertarget{titular-15}{%
\subsection{Titular}\label{titular-15}}

Vivamus libero leo, accumsan non turpis vel, eleifend sodales mauris.
Donec pellentesque, tellus a lacinia vestibulum, nisl lacus dapibus
nibh, id tincidunt turpis erat non nisl. Donec ornare imperdiet metus,
eu sollicitudin risus elementum non. Aliquam metus eros, maximus
dignissim pellentesque nec, venenatis vitae purus. Vivamus laoreet mi
magna, ac malesuada tellus condimentum in. Integer nulla lectus, finibus
sit amet ex nec, finibus molestie diam. Pellentesque gravida ac erat sit
amet tempus.

\bookmarksetup{startatroot}

\hypertarget{references}{%
\chapter*{References}\label{references}}
\addcontentsline{toc}{chapter}{References}

\markboth{References}{References}

\hypertarget{refs}{}
\begin{CSLReferences}{0}{0}
\end{CSLReferences}



\end{document}
